\documentclass[spanish, 10pt,a4paper]{article}
\usepackage[spanish]{babel}
\usepackage[utf8]{inputenc}
\usepackage{textcomp}
\usepackage{hyperref}
\usepackage[pdftex]{graphicx}
\usepackage{epsfig}
\usepackage{amsmath}
\usepackage{hyperref}
\usepackage{amssymb}
\usepackage{color}
\usepackage{graphics}
\usepackage{amsthm}
\usepackage{subcaption}
\usepackage{caratula}
\usepackage{fancyhdr,lastpage}
\usepackage[paper=a4paper, left=1.4cm, right=1.4cm, bottom=1.4cm, top=1.4cm]{geometry}
\usepackage[table]{xcolor} % color en las matrices
\usepackage[font=small,labelfont=bf]{caption} % caption de las figuras en letra mas chica que el texto
\usepackage[ruled,vlined,linesnumbered]{algorithm2e}
\usepackage{listings}
\usepackage{float}
\usepackage{amsfonts}
\usepackage{upgreek}

\color{black}

%%%PAGE LAYOUT%%%
\topmargin = -1.2cm
\voffset = 0cm
\hoffset = 0em
\textwidth = 48em
\textheight = 164 ex
\oddsidemargin = 0.5 em
\parindent = 2 em
\parskip = 3 pt
\footskip = 7ex
\headheight = 20pt
\pagestyle{fancy}
\lhead{Paradigmas y lenguajes de programaci\'on - 2015 C1 - Trabajo Pr\'actico 3} % cambia la parte izquierda del encabezado
\renewcommand{\sectionmark}[1]{\markboth{#1}{}} % cambia la parte derecha del encabezado
\rfoot{\thepage}
\cfoot{}

%spaced sections



%El siguiente paquete permite escribir la caratula facilmente
\hypersetup{
  pdftitle={ 1c2015.PLP.TP2},
  colorlinks,
  citecolor=black,
  filecolor=black,
  linkcolor=black,
  urlcolor=black 
}

\materia{Paradigmas y lenguajes de programaci\'on}

\titulo{Trabajo Pr\'actico 2}

\subtitulo{Programaci\'on l\'ogica.}

\grupo{Grupo: Zamba c\'alculo}

\integrante{Leandro Vega}{698/11}{leandrovega@gmail.com}
\integrante{Ignacio Niesz}{722/10}{ignacio.niesz@gmail.com}
 
\begin{document}
{ \oddsidemargin = 2em
	\headheight = -20pt
	\maketitle	
}
  
	\tableofcontents
	\newpage
	\section{Modelo de mediciones}

\lstset{
  frame=none,
  xleftmargin=2pt,
  stepnumber=1,
  numbers=left,
  numbersep=5pt,
  numberstyle=\ttfamily\tiny\color[gray]{0.3},
  belowcaptionskip=\bigskipamount,
  captionpos=b,
  escapeinside={*'}{'*},
  language=prolog,
  tabsize=2,
  emphstyle={\bf},
  commentstyle=\it,
  stringstyle=\mdseries\rmfamily,
  showspaces=false,
  keywordstyle=\bfseries\rmfamily,
  columns=flexible,
  basicstyle=\small\sffamily,
  showstringspaces=false,
  morecomment=[l]\%,
}


\subsection{Unidad}

\begin{lstlisting}

Object subclass: #Unidad
	instanceVariableNames: ''
	classVariableNames: ''
	poolDictionaries: ''
	category: 'PLP-TP-2015C1'

"-- -- -- -- -- -- -- -- -- -- -- -- -- -- -- -- -- -- "

Unidad class
	instanceVariableNames: ''

escalar
	^UnidadEscalar new.

metro
	^UnidadBasica deNombre:'metro'.

kilogramo
	^UnidadBasica deNombre:'kilogramo'.

segundo
	^UnidadBasica deNombre:'segundo'.

\end{lstlisting}
\vspace{5mm}

\subsubsection{Producto de Unidades}
\begin{lstlisting}
	
Unidad subclass: #ProductoDeUnidades
	instanceVariableNames: 'factores'
	classVariableNames: ''
	poolDictionaries: ''
	category: 'PLP-TP-2015C1'

* anUnidad 
	(anUnidad class = ProductoDeUnidades ) ifTrue: [^ self productoProducto: anUnidad].
	^ anUnidad * self.

productoProducto: aProductoDeUnidades 
	^ ProductoDeUnidades de: ((self factores) addAll: (aProductoDeUnidades factores); yourself).

init: aCollection 
	factores := aCollection.

factores
	^ factores.

printOn: aStream 
	(1 to: (factores size - 1)) do: [ :each | (factores at:each) printOn: aStream . aStream nextPutAll: ' ' ].
	^(factores at:(factores size)) printOn: aStream.

= otraUnidad
	^ (self class = otraUnidad class) and: [self factores asBag = otraUnidad factores asBag ].

"-- -- -- -- -- -- -- -- -- -- -- -- -- -- -- -- -- -- "

ProductoDeUnidades class
	instanceVariableNames: ''!

de: factores 
	factores ifEmpty: [ ^ Unidad escalar ]. 
	(factores size = 1) ifTrue: [ ^ factores first ].
	^ self new init: factores! !

\end{lstlisting}
\vspace{5mm}



\subsubsection{Unidad B\'asica}
\begin{lstlisting}
Unidad subclass: #UnidadBasica
	instanceVariableNames: 'nombre'
	classVariableNames: ''
	poolDictionaries: ''
	category: 'PLP-TP-2015C1'

initNombre: aString 
	nombre := aString.

hash
	^ nombre hash.

nombre
	^ nombre.
	
* anUnidad
	(anUnidad class = UnidadEscalar) ifTrue: [ ^ self productoEscalar: anUnidad ].
	(anUnidad class = UnidadBasica) ifTrue: [^ self productoBasica: anUnidad].
	^ self productoProducto: anUnidad.


productoEscalar: anUnidadEscalar 
	^ self.

productoBasica: anUnidadBasica
	^ ProductoDeUnidades de: (OrderedCollection  with: self with: anUnidadBasica).

productoProducto: aProductoDeUnidades 
	^ ProductoDeUnidades de: ( (aProductoDeUnidades factores) add: self ; yourself).

= otraUnidad
	^ (self class = otraUnidad class) and: [self nombre = otraUnidad nombre].

printOn: aStream 
	^ aStream nextPutAll: nombre.

"-- -- -- -- -- -- -- -- -- -- -- -- -- -- -- -- -- -- "

UnidadBasica class
	instanceVariableNames: ''

deNombre: aString 
	^ self new initNombre: aString.

\end{lstlisting}
\vspace{5mm}

\subsubsection{Unidad Escalar}
\begin{lstlisting}
Unidad subclass: #UnidadEscalar
	instanceVariableNames: ''
	classVariableNames: ''
	poolDictionaries: ''
	category: 'PLP-TP-2015C1'!

* anUnidad
	(anUnidad class = UnidadEscalar) ifTrue: [ ^ self productoEscalar: anUnidad ].
	(anUnidad class = UnidadBasica) ifTrue: [^ self productoBasica: anUnidad].
	^ self productoProducto: anUnidad.

productoEscalar: anUnidadEscalar 
	^ anUnidadEscalar.

productoBasica: anUnidadBasica 
	^ anUnidadBasica .

productoProducto: aProductoDeUnidades 
	^ aProductoDeUnidades.

= otraUnidad
	^ (self class = otraUnidad class).

printOn: aStream 
	^ aStream nextPutAll: 'escalar'.

\end{lstlisting}
\vspace{5mm}

\subsection{Medida}
\begin{lstlisting}
Object subclass: #Medida
	instanceVariableNames: 'cantidad unidad'
	classVariableNames: ''
	poolDictionaries: ''
	category: 'PLP-TP-2015C1'

cantidad: anInteger unidad: anUnidad
	cantidad := anInteger.
	unidad := anUnidad.

cantidad
	^ cantidad.

unidad
	^ unidad.

+ otraMedida
	(self unidad = otraMedida unidad) ifTrue: [^ Medida new cantidad: (self cantidad + otraMedida cantidad) unidad: (self unidad)].
	^ super + otraMedida.

- otro 
	^ self + ((-1) * otro).

* otraMedida 
	^ self cantidad: cantidad * (otraMedida cantidad) unidad: unidad * (otraMedida unidad).

metro
	^ Medida new cantidad: (self cantidad) unidad: ((self unidad) * (Unidad metro)).

kilogramo
	^ Medida new cantidad: (self cantidad) unidad: ((self unidad) * (Unidad kilogramo)).

segundo
	^ Medida new cantidad: (self cantidad) unidad: ((self unidad) * (Unidad segundo)).

= medida2
	^ (self class = medida2 class) and: [(cantidad = medida2 cantidad) and: (unidad = medida2 unidad)].

printOn: aStream
	aStream nextPutAll: cantidad asString. 
	aStream nextPutAll: ' '. 
	^unidad printOn: aStream.

\end{lstlisting}
\vspace{5mm}

\subsection{Number}
\begin{lstlisting}

cantidad
	^self.

unidad
	^Unidad escalar.

kilogramo
	^Medida new cantidad: self unidad: Unidad kilogramo.

metro
	^Medida new cantidad: self unidad: Unidad metro.
	
segundo
	^Medida new cantidad: self unidad: Unidad segundo.

\end{lstlisting}
	\newpage

\end{document}

